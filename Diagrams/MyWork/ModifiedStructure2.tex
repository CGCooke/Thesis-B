\tikzset{
block/.style = {draw, fill=white, rectangle, minimum height=3em, minimum width=3em},
tmp/.style  = {coordinate}, 
sum/.style= {draw, fill=white, circle, node distance=1cm},
input/.style = {coordinate},
output/.style= {coordinate},
pinstyle/.style = {pin edge={to-,thin,black}
}
}

\begin{figure}[!htb]
\centering
\begin{tikzpicture}[auto, node distance=2cm,>=latex']
    \node [input, name=rinput] (rinput) {};
    \node [sum, right of=rinput,node distance = 2cm](Rotator){\Large$+$};
    \node [block, below of =Rotator](VCO){$\frac{1}{S}$};
    
    %PLL Stuff
  
    
    \node at (4,-6) [sum] (sum1PLL) {\Large$+$};
  
    \node [block, above of = sum1PLL](K1PLL){$\frac{K_{1}}{T} + K_{3}$};
    \node [block, right of = sum1PLL](K2PLL){$\frac{K_{2}}{T} + K_{4}$};
    \node [block, right of = K2PLL](ExtraIntegrator){$\frac{1}{S}$};
    
    \node [block, below of = K2PLL](IntegratorPLL){$\frac{1}{S^2}$};
    \node [block, left of = IntegratorPLL](K3PLL){$K_{5}$};
    
     
    
    \node [input, right of = K1PLL,node distance =6cm] (PLLPickoffPoint1) {};
    \node [input, right of = K2PLL,node distance =4cm] (PLLPickoffPoint2) {};
    
    
    %Common
    \draw [->] (rinput) -- node{$\phi_{Input}$}(Rotator);
    \draw [-] (Rotator) -| node{} (PLLPickoffPoint1);
    
    
    \draw [-] (PLLPickoffPoint1) -- node{} (PLLPickoffPoint2);
    \draw [->] (PLLPickoffPoint1) -- node{} (K1PLL);
    \draw [->] (PLLPickoffPoint2) -- node{} (ExtraIntegrator);
    \draw [->] (ExtraIntegrator) -- node{} (K2PLL);
    
    \draw [->] (PLLPickoffPoint2) |- node{} (IntegratorPLL);

    \draw [->] (IntegratorPLL) -- node{} (K3PLL);
   
    
    \draw [->] (K1PLL) -- node{} (sum1PLL);
    \draw [->] (K2PLL) -- node{} (sum1PLL);
    \draw [->] (K3PLL) -- node{} (sum1PLL);

    \draw [->] (sum1PLL) -| node{} (VCO);
    \draw [->] (VCO) -- node{$\phi_{VCO}$} node[pos=0.90] {$-$} (Rotator);
    
    \end{tikzpicture}
\caption{The reduced block diagram, to simplify the diagram, we assume that $K_{VCO}=1$.} 
\label{fig:ModifiedStructure2}
\end{figure}
