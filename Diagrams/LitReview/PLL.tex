\tikzset{
block/.style = {draw, fill=white, rectangle, minimum height=3em, minimum width=3em},
tmp/.style  = {coordinate}, 
sum/.style= {draw, fill=white, circle, node distance=1cm},
input/.style = {coordinate},
output/.style= {coordinate},
pinstyle/.style = {pin edge={to-,thin,black}
}
}

\begin{figure}[!htb]
\centering
\begin{tikzpicture}[auto, node distance=4cm,>=latex']
    \node [input, name=rinput] (rinput) {};
    
    
    \node [block, right of=rinput](PhaseDetector){Phase Detector};
    
    \node [block, right of=PhaseDetector] (LoopFilter) {Loop Filter};
    
    \node [block, below of = PhaseDetector,node distance = 2
    cm](VCO){VCO};
    
    \draw [->] (rinput) -- node{$\theta_{i}$} (PhaseDetector);
    \draw [->] (PhaseDetector) -- node{$\phi$} (LoopFilter);
    \draw [->] (LoopFilter) |- node{$V_{c}$} (VCO);
    \draw [->] (VCO) -- node{$\theta_{o}$} (PhaseDetector);
    
    \end{tikzpicture}
\caption{An abstracted diagram of a typical PLL.} 
\label{fig:PLLAnalog}
\end{figure}
