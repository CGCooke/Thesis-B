\begin{figure}[!htb]
\centering

\begin{tikzpicture}[->,>=stealth',shorten >=1pt,auto,node distance=8cm,semithick]
  \tikzstyle{every state}=[fill=red,draw=none,text=white]

  \node[initial,state] (A)                    {$A$};
  \node[state]         (B) [above right of=A] {$B$};
  \node[state]         (C) [below right of=B] {$C$};
  \node[state]         (D) [below right of=A] {$D$};

  \path (A) edge    [bend left]          node {$\alpha \wedge \beta$} (B)
        
        (B) edge [bend left]  node {} (A)
        
        (B) edge  [bend left] node {$\alpha \wedge \beta \wedge \gamma \wedge \delta$} (C)
        
        (B) edge [bend left] node [near end] {$\alpha \wedge \beta \wedge \gamma \wedge \delta \wedge \epsilon$} (D)
        
        (C) edge              node {} (A)
        (C) edge  [bend left] node [near start,above=10pt] {$\alpha \wedge \beta$} (B)
        
        
        (C) edge [bend left] node {$\alpha \wedge \beta \wedge \gamma \wedge \delta \wedge \epsilon$} (D)
        (D) edge [bend left]  node [near start] {$\alpha \wedge \beta$} (B)
        (D) edge [bend left] node {} (A);
        
\end{tikzpicture}
\caption{State transition diagram. The key to the symbols used for the transition conditions can be found in table \ref{tab:StateTransitionDiagramKey}.} \label{fig:StateTransitionDiagram}
\end{figure}