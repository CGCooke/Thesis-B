\begin{center}
\epsfig{file=../PDFs/Arms-vl.eps,silent,width=10cm}\\[0.5cm]
\textbf{\large SCHOOL OF ELECTRICAL ENGINEERING\\
AND TELECOMMUNICATION}\\[2cm]
{\addtolength{\baselineskip}{0.5cm}
\textbf{\Huge
High Dynamics GNSS Receiver}\\[0.5cm]
}
{\Large by}\\[0.5cm]
\textit{\huge
Cameron Cooke} \\[1.5cm]
{\Large
Thesis B report submitted as a requirement for the degree\\
Bachelor of Engineering (Electrical Engineering)\\[2ex]
\vfill
Submitted: \today\hfill
Student ID: z3288185\\[-1.5ex]
Supervisor: Professor Andrew Dempster\hfill
Topic ID: AGD67\\
\vspace*{-1cm}
}
\end{center}


\begin{abstract}
A \ac{GNSS} allows a receiver to be accurately and reliably positioned, and are widely used for guidance, navigation and control.
The ability of a receiver to provide a navigation solution
depends on the ability of it's tracking loops to track the 
frequency and phase of the incoming signal. Excessive dynamics 
experienced during the launch, stage separation and re-entry phases
of space flight place significant stress on these tracking loops. 
This prevents the signal from being tracked, and the receiver from
computing a position. Orbital launch systems typically use an
\ac{INS} for guidance, however the accuracy
of these systems drifts over time, and they are inferior to \ac{GNSS}
receivers in terms of power, weight, size and cost. Accordingly, there is a
need for developing a \ac{GNSS} receiver suitable for guiding 
space craft during periods of high dynamics. The immediate objective of this thesis
is to improve the high dynamics performance of a space-qualified 
\ac{GNSS} receiver called \ac{NAMURU}, developed by the \ac{ACSER} at The University of 
New South Wales (UNSW). A rigorous literature review and a thorough analysis of the current implementation of the receiver was conducted.
In order to gain a more sophisticated understanding of the current receiver operation,
a software model was constructed, allowing detailed Monte Carlo simulations.
The software simulation was verified by operation both on synthetic data generated by a \ac{GNSS}
simulator, as well as real world data, collected from an aircraft during flight.
\end{abstract}


ACKNOWLEDGEMENTS\\
I acknowledge the contribution of all members of the ACSER team, without whom this work would not have been possible. Specifically:\\
\begin{itemize}
\item{Professor Andrew Dempster}\\
\item{Dr Eamonn Glennon}\\
\item{Dr Joon Wayn Cheong}\\

\end{itemize}
In addition, I thank Dr Arash Khatamianfar for his insight into control systems.
