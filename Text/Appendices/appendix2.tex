\addcontentsline{toc}{chapter}{Appendix 2}\label{app2}
\chapter{Laplace Analysis}

\emph{"All models are wrong, but some are useful" -George E. P. Box}


\begin{table}[!htb]
\resizebox{\textwidth}{!}{%
\begin{tabular}{@{}llll@{}}
\toprule
Name     & Physical interpretation & Time function    & Laplace transform \\ \midrule
Step     & Constant position       & $1$              & $\frac{1}{S}$     \\\addlinespace[5pt]
Ramp     & Constant velocity       & $t$              & $\frac{1}{S^2}$   \\\addlinespace[5pt]
Parabola & Constant acceleration   & $\frac{1}{2}t^2$ & $\frac{1}{S^3}$  \\\addlinespace[5pt]
\end{tabular}
}
\caption{Test inputs for evaluating errors \cite{Nise}}
\label{tab:LaplaceTestInputs}
\end{table}


\tikzset{
block/.style = {draw, fill=white, rectangle, minimum height=3em, minimum width=3em},
tmp/.style  = {coordinate}, 
sum/.style= {draw, fill=white, circle, node distance=1cm},
input/.style = {coordinate},
output/.style= {coordinate},
pinstyle/.style = {pin edge={to-,thin,black}
}
}

\begin{figure}[!htb]
\centering
\begin{tikzpicture}[auto, node distance=2cm,>=latex']
    \node [input, name=rinput] (rinput) {};
    \node [sum, right of=rinput,node distance = 2
    cm](Rotator){\Large$+$};
    
    %FLL Stuff
    \node [block, right of=Rotator,node distance=2cm] (DiscriminatorFLL) {$D_{FLL}$};
    
    \node [block, below of = DiscriminatorFLL,node distance =6cm](K1FLL){$K_{1}$};
    
    \node [block, below of = Rotator](VCO){VCO};
    \node [block, below of = VCO](VCOIntegrator){$\displaystyle \int$};
    
    
    %Common
    \draw [->] (rinput) -- node{$\phi_{Input}$} (Rotator);
    
    %FLL Stuff
    \draw [->] (Rotator) -- node{} (DiscriminatorFLL);
    \draw [->] (DiscriminatorFLL) -- node{} (K1FLL);
    
    \draw [->] (K1FLL) -| node{} (VCOIntegrator);
    \draw [->] (VCOIntegrator) -- node{} (VCO);
    \draw [->] (VCO) -- node{$\phi_{VCO}$} node[pos=0.90] {$-$}  (Rotator);
    
    \end{tikzpicture}
\caption{An analog model of tracking loop for state A.} \label{fig:StateA_Analog}
\end{figure}

\tikzset{
block/.style = {draw, fill=white, rectangle, minimum height=3em, minimum width=3em},
tmp/.style  = {coordinate}, 
sum/.style= {draw, fill=white, circle, node distance=1cm},
input/.style = {coordinate},
output/.style= {coordinate},
pinstyle/.style = {pin edge={to-,thin,black}
}
}

\begin{figure}[!htb]
\centering
\begin{tikzpicture}[auto, node distance=2cm,>=latex']
    \node [input, name=rinput] (rinput) {};
    \node [sum, right of=rinput,node distance = 2
    cm](Rotator){\Large$+$};
    
    %FLL Stuff
    \node [block, right of=Rotator,node distance=3cm] (DiscriminatorFLL) {$S$};
    \node [block, below of = DiscriminatorFLL,node distance =6cm](K1FLL){$K_{1}$};
    \node [block, below of = Rotator](VCO){$\frac{1}{S}$};
    \node [block, below of = VCO](VCOGain){$K_{VCO}$};
    \node [block, below of = VCOGain](VCOIntegrator){$\frac{1}{S}$};
    
    
    %Common
    \draw [->] (rinput) -- node{$\phi_{Input}$} (Rotator);
    
    %FLL Stuff
    \draw [->] (Rotator) -- node{$\phi_{Error}$} (DiscriminatorFLL);
    \draw [->] (DiscriminatorFLL) -- node{$f_{Error}$} (K1FLL);
  
    \draw [->] (K1FLL) -- node{} (VCOIntegrator);
    \draw [->] (VCOIntegrator) -- node{} (VCOGain);
    \draw [->] (VCOGain) -- node{} (VCO);
    
    
    \draw [->] (VCO) -- node{$\phi_{VCO}$} node[pos=0.90] {$-$}  (Rotator);
    
    

    \end{tikzpicture}
\caption{A Laplace domain model for the state A tracking loop. The VCO has been replaced with a gain and an integrator. } \label{fig:StateA_Laplace}
\end{figure}

\tikzset{
block/.style = {draw, fill=white, rectangle, minimum height=3em, minimum width=3em},
tmp/.style  = {coordinate}, 
sum/.style= {draw, fill=white, circle, node distance=1cm},
input/.style = {coordinate},
output/.style= {coordinate},
pinstyle/.style = {pin edge={to-,thin,black}
}
}

\begin{figure}[!htb]
\centering
\begin{tikzpicture}[auto, node distance=2cm,>=latex']
    \node [input, name=rinput] (rinput) {};
    \node [sum, right of=rinput,node distance = 2
    cm](Rotator){\Large$+$};
    
     \node [input, right of = Rotator] (RotatorPickoffPoint) {};
     
    
    %FLL Blocks
    \node [block, right of=RotatorPickoffPoint,node distance=3cm] (DiscriminatorFLL) {$K_{DFLL}$};
    
    %Integrator
    \node [input,right of=DiscriminatorFLL] (Point1FLL) {};
    \node [block, below right of=Point1FLL,node distance=2cm] (Delay1FLL){$Z^{-1}$};
    \node [sum, right of=Point1FLL,node distance=3cm] (sumIntegrator1FLL) {\Large$+$};
    
    \node [block, right of=sumIntegrator1FLL,node distance=2cm] (SamplingTime){$\frac{1}{T}$};  
    
    %Proportional FLL 
    \node [block, below of = DiscriminatorFLL,node distance =7cm](K1FLL){$K_{1}$};
    \node [input, below of = K1FLL,node distance = 2cm](Point2FLL){};
    
    

    \node [sum, left of = K1FLL,node distance =5cm](sumCommon){\Large$+$};
    
    
    \node [input, above of = sumCommon,node distance=3cm](PointNCO){};
    \node [block, above left of = sumCommon](delayNCO){\Large$Z^{-1}$};
    
    \node [block, above of = PointNCO](NCO){NCO};
    
    
    
    \draw [->] (rinput) -- node{$\phi_{Input}$} (Rotator);
    \draw [->] (PointNCO) -- node{} (NCO);
    \draw [->] (NCO) -- node{$\phi_{NCO}$} node[pos=0.90] {$-$} (Rotator);
    
    
    
    \draw [->] (PointNCO) -| node{} (delayNCO);
    \draw [->] (delayNCO) |- node{} (sumCommon);
    
    
    
    %FLL Connections
    \draw [-] (Rotator) -- node{$\phi_{Error}$} (RotatorPickoffPoint);
    
    \draw [->] (RotatorPickoffPoint) -- node{} (DiscriminatorFLL);
    
    
    \draw [->] (DiscriminatorFLL) -- node{} (sumIntegrator1FLL);
    
    %FLL Discriminator Integrator
    \draw [->] (Point1FLL) |- node{} (Delay1FLL);
    \draw [->] (Delay1FLL) -| node[pos=0.95]{$-$} (sumIntegrator1FLL);
    \draw [->] (sumIntegrator1FLL) -- node{} (SamplingTime);
    
    
    %Loop Filter Integrator
    \draw [->] (SamplingTime) |- node[pos=0.75]{$f_{error}$} (K1FLL);
    \draw [->] (K1FLL) -- node{} (sumCommon);
    \draw [-] (sumCommon) -| node{} (PointNCO);
    
    
    %% Boxing and labelling noise shapers
	%\draw [color=gray,thick](-0.5,-3) rectangle (9,1);
	%\node at (-0.5,1) [above=5mm, right=0mm] {\textsc{first-order noise shaper}};
	%\draw [color=gray,thick](-0.5,-9) rectangle (12.5,-5);
	%\node at (0,0) {\textsc{second-order noise shaper}};
    
   
    
    \end{tikzpicture}
\caption{A digital model of state A.} \label{fig:StateA_Digital}
\end{figure}



In order to have an understanding of the behaviour of the tracking loops, we must first model them. By modelling the tracking loops in the Laplace domain, we are able to get a rough idea of the behaviour of the loops. 

\section{State A}
\subsection{Open loop transfer function}
In figures \ref{fig:StateA_Analog} and \ref{fig:StateA_Laplace}, you can see analog models of the FLL used in state A. $K_1$ has a value of 0.125. T = 4ms, $K_{VCO} = 1$

Then we have the open loop transfer function of 

\begin{equation}
S \times \frac{1}{T} \times K_1 \times \frac{1}{S} \times  
K_{VCO} \times \frac{1}{S}
\end{equation}

or 
\begin{equation}
\frac{K_1 K_{VCO}}{T S}
\end{equation}

\subsection{Closed loop transfer function}
Converting to a closed loop transfer function :

\begin{equation}
\frac{1}{1+\frac{K_1 K_{VCO}}{T S}}
\end{equation}


\begin{equation}
\frac{S}{S+\frac{K_1 K_{VCO}}{T}}
\end{equation}

equals 
\begin{equation}
\frac{S}{S+31.25}
\end{equation}

\subsection{Step response}
A step input in terms of phase is in the Laplace domain is characterised as  $\frac{1}{S}$

Hence the system response to a step in phase is
\begin{equation}
\frac{1}{S+31.25}
\end{equation}


In the time domain,

\begin{equation}
 error(t) =  e^{-31.25t}
\end{equation}
Hence the system rapidly adjusts to the step input. 

\subsection{Ramp response}

For a ramp in phase, IE, a constant frequency, then we have $\frac{1}{S^2}$

This gives 

\begin{equation}
\frac{1}{S^2+31.25S}
\end{equation}


\begin{equation}
\frac{0.032}{S} - \frac{0.032}{S+32.25} 
\end{equation}



Which is: 
\begin{equation}
error(t) =  0.0320 -0.0320e^{-31.25t}
\end{equation}

In the time domain.

Hence for constant frequency there will be a small phase error.

\subsection{Parabolic response}
For ramp input in frequency, or a parabolic input in phase, we will have 

\begin{equation}
\frac{1}{S^3}
\end{equation}

Which gives 

\begin{equation}
\frac{1}{S^3+31.25S^2}
\end{equation}

\begin{equation}
\frac{0.032}{S^2} - \frac{0.001024}{S} + \frac{0.001024}{S+31.25}
\end{equation}

Which is 

\begin{equation}
error(t) =  0.032t - 0.001024 + 0.001024 e^{-31.25}
\end{equation}

Hence there will be an increasing phase error with time, or an constant frequency error with time. 

If we know the frequency/phase rate, then we can work out the constant frequency error at this point.

\section{State B \& C}
\tikzset{
block/.style = {draw, fill=white, rectangle, minimum height=3em, minimum width=3em},
tmp/.style  = {coordinate}, 
sum/.style= {draw, fill=white, circle, node distance=1cm},
input/.style = {coordinate},
output/.style= {coordinate},
pinstyle/.style = {pin edge={to-,thin,black}
}
}

\begin{figure}[!htb]
\centering
\begin{tikzpicture}[auto, node distance=2cm,>=latex']
    \node [input, name=rinput] (rinput) {};
    \node [sum, right of=rinput,node distance = 2
    cm](Rotator){\Large$+$};
    \node [input, right of = Rotator] (RotatorPickoffPoint) {};
    
    
    %FLL Stuff
    \node [block, right of=RotatorPickoffPoint,node distance=4cm] (DiscriminatorFLL) {$D_{FLL}$};
    \node [block, right of=DiscriminatorFLL] (SamplingTime){$\frac{1}{T}$};
    \node [block, below of = DiscriminatorFLL,node distance =6cm](K1FLL){$K_{1}$};
    
    
    \node [sum, left of = K1FLL,node distance =6cm](sumCommon){\Large$+$};
    
    \node [block, above of = sumCommon](VCOIntegrator){$\displaystyle \int$};
    
    
    \node [block, above of = VCOIntegrator](VCO){VCO};
    \node [input, right of = K1FLL] (FLLPickoffPoint) {};
    
    
    
    %PLL Stuff
    \node [block, above of=DiscriminatorFLL] (DiscriminatorPLL) {$D_{PLL}$};
    
    \node [block, below of = K1FLL](diffPLL){$\frac{d}{dt}$};
    \node [block, left of = diffPLL](K1PLL){$K_2$};
    \node [block, below of = diffPLL](K2PLL){$K_3$};
    \node [sum, left of=K2PLL,node distance=2cm] (sum1PLL) {\Large$+$};
    \node [output, right of=DiscriminatorPLL, node distance=4cm] (output) {};
     
    
    \node [input, right of = diffPLL,node distance =4cm] (PLLPickoffPoint1) {};
    \node [input, right of = K2PLL,node distance =4cm] (PLLPickoffPoint2) {};
    
    
    %Common
    \draw [->] (rinput) -- node{$\phi_{Input}$} (Rotator);
    
    %FLL Stuff
    \draw [-] (Rotator) -- node{$\phi_{Error}$} (RotatorPickoffPoint);
    
    \draw [->] (RotatorPickoffPoint) -- node{} (DiscriminatorFLL);
    
    
    
    \draw [->] (DiscriminatorFLL) -- node{} (SamplingTime);
    
    \draw [-] (SamplingTime) |- node{} (FLLPickoffPoint);
    \draw [->] (FLLPickoffPoint) -- node{} (K1FLL);
    
    \draw [->] (K1FLL) -- node{} (sumCommon);
    
    %PLL Stuff
    \draw [->] (RotatorPickoffPoint) |- node{} (DiscriminatorPLL);
    
    \draw [-] (DiscriminatorPLL) -- node [name=y] {}(output);
    
    \draw [-] (output) -- node{} (PLLPickoffPoint1);
    \draw [-] (PLLPickoffPoint1) -- node{} (PLLPickoffPoint2);
    \draw [->] (PLLPickoffPoint1) -- node{} (diffPLL);
    \draw [->] (PLLPickoffPoint2) -- node{} (K2PLL);
    

    \draw [->] (diffPLL) -- node{} (K1PLL);
    
    \draw [->] (K1PLL) -- node{} (sum1PLL);
    \draw [->] (K2PLL) -- node{} (sum1PLL);
    
    \draw [->] (sum1PLL) -| node{} (sumCommon);
    \draw [->] (sumCommon) -- node{} (VCOIntegrator);
    \draw [->] (VCOIntegrator) -- node{} (VCO);
    \draw [->] (VCO) -- node{$\phi_{VCO}$} node[pos=0.90] {$-$} (Rotator);
    
    

    \end{tikzpicture}
\caption{An analog model of state B \& C.} \label{fig:StateB_Analog}
\end{figure}

\tikzset{
block/.style = {draw, fill=white, rectangle, minimum height=3em, minimum width=3em},
tmp/.style  = {coordinate}, 
sum/.style= {draw, fill=white, circle, node distance=1cm},
input/.style = {coordinate},
output/.style= {coordinate},
pinstyle/.style = {pin edge={to-,thin,black}
}
}

\begin{figure}[!htb]
\centering
\begin{tikzpicture}[auto, node distance=2cm,>=latex']
    \node [input, name=rinput] (rinput) {};
    \node [sum, right of=rinput,node distance = 2
    cm](Rotator){\Large$+$};
    
     \node [input, right of = Rotator] (RotatorPickoffPoint) {};
     
    
    %FLL Blocks
    \node [block, right of=RotatorPickoffPoint,node distance=3cm] (DiscriminatorFLL) {$K_{DFLL}$};
    
    %Integrator
    \node [input,right of=DiscriminatorFLL] (Point1FLL) {};
    \node [block, below right of=Point1FLL,node distance=2cm] (Delay1FLL){$Z^{-1}$};
    \node [sum, right of=Point1FLL,node distance=3cm] (sumIntegrator1FLL) {\Large$+$};
    
    \node [block, right of=sumIntegrator1FLL,node distance=2cm] (SamplingTime){$\frac{1}{T}$};  
    
    %Proportional FLL 
    \node [block, below of = DiscriminatorFLL,node distance =7cm](K1FLL){$K_{1}$};
    \node [input, below of = K1FLL,node distance = 2cm](Point2FLL){};
    
    
    
    
    %PLL Blocks
    \node [block, above of=DiscriminatorFLL] (DiscriminatorPLL) {$K_{DPLL}$};
    
    %Differentiator PLL
    \node [sum, below of = K1FLL,node distance =3cm](sumDifferentiator1PLL) {\Large$+$};
    \node [input, right of = sumDifferentiator1PLL,node distance = 3cm] (Point2PLL) {};
    \node [block, below right of=sumDifferentiator1PLL,node distance=2cm] (Delay2PLL){$Z^{-1}$};
    
    \node [input, right of = Point2PLL,node distance = 5cm] (PLLPickoffPoint1) {};
    
    
    
    %Proportional PLL
    \node [block, left of = sumDifferentiator1PLL](K1PLL){$K_2$};
    \node [block, below of = sumDifferentiator1PLL,node distance = 3cm](K2PLL){$K_3$};
    
    
    \node [input, right of = K2PLL,node distance = 8cm] (PLLPickoffPoint2) {};
    
    
    \node [sum, left of=K2PLL,node distance=2cm] (sum1PLL) {\Large$+$};

    %Common
    \node [output, right of=DiscriminatorPLL, node distance=8cm] (output) {};
    


    \node [sum, left of = K1FLL,node distance =5cm](sumCommon){\Large$+$};
    
    
    \node [input, above of = sumCommon,node distance=3cm](PointNCO){};
    \node [block, above left of = sumCommon](delayNCO){\Large$Z^{-1}$};
    
    \node [block, above of = PointNCO](NCO){NCO};
    
    
    
    \draw [->] (rinput) -- node{$\phi_{Input}$} (Rotator);
    \draw [->] (PointNCO) -- node{} (NCO);
    \draw [->] (NCO) -- node{$\phi_{NCO}$} node[pos=0.90] {$-$} (Rotator);
    
    \draw [->] (PointNCO) -| node{} (delayNCO);
    \draw [->] (delayNCO) |- node{} (sumCommon);
    
    %FLL Connections
    \draw [-] (Rotator) -- node{$\phi_{Error}$} (RotatorPickoffPoint);
    
    \draw [->] (RotatorPickoffPoint) -- node{} (DiscriminatorFLL);
    
    
    \draw [->] (DiscriminatorFLL) -- node{} (sumIntegrator1FLL);
    
    %FLL Discriminator Integrator
    \draw [->] (Point1FLL) |- node{} (Delay1FLL);
    \draw [->] (Delay1FLL) -| node[pos=0.95]{$-$} (sumIntegrator1FLL);
    \draw [->] (sumIntegrator1FLL) -- node{} (SamplingTime);
    
    
    %Loop Filter Integrator
    
    
    \draw [->] (SamplingTime) |- node{} (K1FLL);
    \draw [->] (K1FLL) -- node{} (sumCommon);
    
    
    %PLL Connections
    \draw [->] (RotatorPickoffPoint) |- node[]{} (DiscriminatorPLL);
    
    \draw [-] (DiscriminatorPLL) -- node [name=y] {}(output);
    
    \draw [-] (output) -- node{} (PLLPickoffPoint1);
    \draw [->] (PLLPickoffPoint1) -- node{} (sumDifferentiator1PLL);
    \draw [-] (PLLPickoffPoint1) -- node{} (PLLPickoffPoint2);
    
    \draw [->] (PLLPickoffPoint2) -- node{} (K2PLL);
    
    \draw [->] (sumDifferentiator1PLL) -- node{} (K1PLL);
    
    \draw [->] (Point2PLL) |- node{} (Delay2PLL);
    \draw [->] (Delay2PLL) -| node[pos=0.95]{$-$}  (sumDifferentiator1PLL);
    
    
    
    \draw [->] (K1PLL) -- node{} (sum1PLL);
    \draw [->] (K2PLL) -- node{} (sum1PLL);
    
    \draw [->] (sum1PLL) -| node{} (sumCommon);
    \draw [-] (sumCommon) -| node{} (PointNCO);
    
    
    %% Boxing and labelling noise shapers
	%\draw [color=gray,thick](-0.5,-3) rectangle (9,1);
	%\node at (-0.5,1) [above=5mm, right=0mm] {\textsc{first-order noise shaper}};
	%\draw [color=gray,thick](-0.5,-9) rectangle (12.5,-5);
	%\node at (0,0) {\textsc{second-order noise shaper}};
    
   
    
    \end{tikzpicture}
\caption{A digital model of state B \& C.} \label{fig:CompleteDigitalDiagram}
\end{figure}


\tikzset{
block/.style = {draw, fill=white, rectangle, minimum height=3em, minimum width=3em},
tmp/.style  = {coordinate}, 
sum/.style= {draw, fill=white, circle, node distance=1cm},
input/.style = {coordinate},
output/.style= {coordinate},
pinstyle/.style = {pin edge={to-,thin,black}
}
}

\begin{figure}[!htb]
\centering
\begin{tikzpicture}[auto, node distance=2cm,>=latex']
    \node [input, name=rinput] (rinput) {};
    \node [sum, right of=rinput,node distance = 2
    cm](Rotator){\Large$+$};
    \node [input, right of = Rotator] (RotatorPickoffPoint) {};
    
    
    %FLL Stuff
    \node [block, right of=RotatorPickoffPoint,node distance=4cm] (DiscriminatorFLL) {$S$};
    
    \node [block, below of = DiscriminatorFLL,node distance =8cm](K1FLL){$K_{1}$};
    
    
    \node [sum, left of = K1FLL,node distance =6cm](sumCommon){\Large$+$};
    
    \node [block, above of = sumCommon](VCOIntegrator){$\frac{1}{S}$};
    
    \node [block, above of = VCOIntegrator](VCOGain){$K_{VCO}$};
    
    \node [block, above of = VCOGain](VCO){$\frac{1}{S}$};
    
    
    
    
    \node [input, right of = K1FLL] (FLLPickoffPoint) {};
    
    %PLL Stuff
    \node [block, above of=DiscriminatorFLL] (DiscriminatorPLL) {$1$};
    
    \node [block, below of = K1FLL](diffPLL){$S$};
    \node [block, left of = diffPLL](K1PLL){$K_2$};
    \node [block, below of = diffPLL](K2PLL){$K_3$};
    \node [sum, left of=K2PLL,node distance=2cm] (sum1PLL) {\Large$+$};
    \node [output, right of=DiscriminatorPLL, node distance=4cm] (output) {};
     
    \node [input, right of = diffPLL,node distance =4cm] (PLLPickoffPoint1) {};
    \node [input, right of = K2PLL,node distance =4cm] (PLLPickoffPoint2) {};
    
    %Common
    \draw [->] (rinput) -- node{$\phi_{Input}$} (Rotator);
    
    %FLL Stuff
    \draw [-] (Rotator) -- node{$\phi_{Error}$} (RotatorPickoffPoint);
    
    \draw [->] (RotatorPickoffPoint) -- node{} (DiscriminatorFLL);
    
    \draw [->] (DiscriminatorFLL)  -- node{} (K1FLL);
    
    \draw [->] (K1FLL) -- node{} (sumCommon);
    
    %PLL Stuff
    \draw [->] (RotatorPickoffPoint) |- node{} (DiscriminatorPLL);
    
    \draw [-] (DiscriminatorPLL) -- node [name=y] {}(output);
    
    \draw [-] (output) -- node{} (PLLPickoffPoint1);
    \draw [-] (PLLPickoffPoint1) -- node{} (PLLPickoffPoint2);
    \draw [->] (PLLPickoffPoint1) -- node{} (diffPLL);
    \draw [->] (PLLPickoffPoint2) -- node{} (K2PLL);
    
    \draw [->] (diffPLL) -- node{} (K1PLL);
    
    \draw [->] (K1PLL) -- node{} (sum1PLL);
    \draw [->] (K2PLL) -- node{} (sum1PLL);
    
    \draw [->] (sum1PLL) -| node{} (sumCommon);
    \draw [->] (sumCommon) -- node{} (VCOIntegrator);
    \draw [->] (VCOIntegrator) -- node{} (VCOGain);
    \draw [->] (VCOGain) -- node{} (VCO);
    
    \draw [->] (VCO) -- node{$\phi_{VCO}$} node[pos=0.90] {$-$} (Rotator);

    \end{tikzpicture}
\caption{A Laplace domain model of state B \& C.} \label{fig:StateB_Laplace}
\end{figure}

\subsection{Open loop transfer function}

Assuming $K_{VCO} = 1$, 

Then we have the open loop transfer function of 

\begin{equation}
(S \times \frac{1}{T} \times K_1 + S \times K_2 +K_3 ) \times \frac{1}{S} \times K_{VCO} \times \frac{1}{S}
\end{equation}


\begin{equation}
\frac{1}{S} (\frac{K_1}{T} + K_2 +  \frac{K_3}{S})
\end{equation}

 
\subsection{Closed loop transfer function}
This results in a close loop transfer function of 

\begin{equation}
\frac{1}{1+\frac{1}{S} (\frac{K_1}{T} + K_2 +  \frac{K_3}{S})}
\end{equation}



\begin{equation}
\frac{S^2}{S^2 + (\frac{K_1}{T} + K_2)S + K_3}
\end{equation}

\begin{framed}
\begin{align}
FLL_{BW} &=1\\
PLL_{BW} &=18\\
K_1 &=  0.04\\
K_2 &= 26.68\\
K_3 &=  2.848
\end{align}
\end{framed}

\begin{equation}
\frac{S^2}{S^2 + (\frac{0.04}{0.04} +  26.68)S + 2.848}
\end{equation}

\begin{equation}
\frac{S^2}{S^2 + 27.68 S + 2.848}
\end{equation}

\subsection{Step response}

For an input of $\frac{1}{S}$, we have 
\begin{equation}
\frac{S}{S^2 + 27.68 S + 2.848}
\end{equation}

Or 

\begin{equation}
\frac{-0.003758}{S+0.1033} + \frac{1.00376}{S+27.58}
\end{equation}

Which is

\begin{equation}
error(t) =  -0.003758 e^{-0.103} + 1.00376 e^{-27.58}
\end{equation}


\subsection{Ramp response}

For an input of $\frac{1}{S^2}$, we have 

\begin{equation}
\frac{1}{S^2 + 27.683 S + 2.848}
\end{equation}

or 

\begin{equation}
\frac{0.0364}{S+0.1033} -\frac{0.0364}{S+27.58}
\end{equation}

which is 

\begin{equation}
error(t) =  0.0364e^{-0.1033} - 0.0364e^{-27.58}
\end{equation}


\subsection{Parabolic response}

For an input of $\frac{1}{S^3}$, we have 
\begin{equation}
\frac{1}{S^3 +27.683 S^2 + 2.848 S}
\end{equation}


or 

\begin{equation}
\frac{0.3511}{S}-\frac{0.3524}{0.1033+S}
+\frac{0.001320}{27.58+S}
\end{equation}

which is

\begin{equation}
error(t) = 0.3511 + 0.352 e^{-0.1032} + 0.00132 e^{-27.6}
\end{equation}

\section{State D}
\tikzset{
block/.style = {draw, fill=white, rectangle, minimum height=3em, minimum width=3em},
tmp/.style  = {coordinate}, 
sum/.style= {draw, fill=white, circle, node distance=1cm},
input/.style = {coordinate},
output/.style= {coordinate},
pinstyle/.style = {pin edge={to-,thin,black}
}
}

\begin{figure}[!htb]
\centering
\begin{tikzpicture}[auto, node distance=2cm,>=latex']
    \node [input, name=rinput] (rinput) {};
    \node [sum, right of=rinput,node distance = 2
    cm](Rotator){\Large$+$};
    \node [input, right of = Rotator] (RotatorPickoffPoint) {};
     
    
    
    %FLL Stuff
    \node [block, right of=Rotator,node distance=4cm] (DiscriminatorFLL) {$D_{FLL}$};
    
    \node [block, below of = DiscriminatorFLL,node distance =6cm](K1FLL){$K_{1}$};
    \node [block, below of = K1FLL](IntegratorFLL){$\displaystyle \int$};
    \node [block, left of = IntegratorFLL](K2FLL){$K_{2}$};
    \node [sum, left of=K1FLL,node distance=2cm] (sum1FLL) {\Large$+$};
    \node [sum, left of = sum1FLL,node distance =2cm](sumCommon){\Large$+$};
    
    \node [block, above of = sumCommon](VCOIntegrator){$\displaystyle \int$};
    \node [block, above of = VCOIntegrator](VCO){VCO};
    
    
    \node [input, right of = K1FLL] (FLLPickoffPoint) {};
    
    
    
    %PLL Stuff
    \node [block, above of=DiscriminatorFLL] (DiscriminatorPLL) {$D_{PLL}$};
    \node [block, below of = IntegratorFLL](diffPLL){$\frac{d}{dt}$};
    \node [block, left of = diffPLL](K1PLL){$K_{3}$};
    \node [block, below of = diffPLL](K2PLL){$K_{4}$};
    \node [block, below of = K2PLL](IntegratorPLL){$\displaystyle \int$};
    \node [block, left of = IntegratorPLL](K3PLL){$K_{5}$};
    \node [sum, left of=K2PLL,node distance=2cm] (sum1PLL) {\Large$+$};
    \node [output, right of=DiscriminatorPLL, node distance=4cm] (output) {};
     
    
    \node [input, right of = diffPLL,node distance =4cm] (PLLPickoffPoint1) {};
    \node [input, right of = K2PLL,node distance =4cm] (PLLPickoffPoint2) {};
    
    
    %Common
    \draw [->] (rinput) -- node{$\phi_{Input}$}(Rotator);
    
    %FLL Stuff
    \draw [-] (Rotator) -- node{$\phi_{Error}$} (RotatorPickoffPoint);
    \draw [->] (RotatorPickoffPoint) -- node{} (DiscriminatorFLL);
    
    
    \draw [-] (DiscriminatorFLL) -| node{} (FLLPickoffPoint);
    \draw [->] (FLLPickoffPoint) -- node{} (K1FLL);
    \draw [->] (FLLPickoffPoint) |- node{} (IntegratorFLL);
    \draw [->] (IntegratorFLL) -- node{} (K2FLL);
    
    \draw [->] (K1FLL) -- node{} (sum1FLL);
    \draw [->] (K2FLL) -- node{} (sum1FLL);
    \draw [->] (sum1FLL) -- node{} (sumCommon);
    
    
    %PLL Stuff
    \draw [->] (RotatorPickoffPoint) |- node{} (DiscriminatorPLL);
    
    \draw [-] (DiscriminatorPLL) -- node [name=y] {}(output);
    
    \draw [-] (output) -- node{} (PLLPickoffPoint1);
    \draw [-] (PLLPickoffPoint1) -- node{} (PLLPickoffPoint2);
    \draw [->] (PLLPickoffPoint1) -- node{} (diffPLL);
    \draw [->] (PLLPickoffPoint2) -- node{} (K2PLL);
    \draw [->] (PLLPickoffPoint2) |- node{} (IntegratorPLL);
   

    \draw [->] (IntegratorPLL) -- node{} (K3PLL);
    \draw [->] (diffPLL) -- node{} (K1PLL);
    
    \draw [->] (K1PLL) -- node{} (sum1PLL);
    \draw [->] (K2PLL) -- node{} (sum1PLL);
    \draw [->] (K3PLL) -- node{} (sum1PLL);
    
    \draw [->] (sum1PLL) -| node{} (sumCommon);
    \draw [->] (sumCommon) -- node{} (VCOIntegrator);
    \draw [->] (VCOIntegrator) -- node{} (VCO);
    \draw [->] (VCO) -- node{$\phi_{VCO}$} node[pos=0.90] {$-$} (Rotator);
    
    \end{tikzpicture}
\caption{An analog model of state D.} \label{fig:StateD_Analog}
\end{figure}

\tikzset{
block/.style = {draw, fill=white, rectangle, minimum height=3em, minimum width=3em},
tmp/.style  = {coordinate}, 
sum/.style= {draw, fill=white, circle, node distance=1cm},
input/.style = {coordinate},
output/.style= {coordinate},
pinstyle/.style = {pin edge={to-,thin,black}
}
}

\begin{figure}[!htb]
\centering
\begin{tikzpicture}[auto, node distance=2cm,>=latex']
    \node [input, name=rinput] (rinput) {};
    \node [sum, right of=rinput,node distance = 2
    cm](Rotator){\Large$\times$};
    
     \node [input, right of = Rotator] (RotatorPickoffPoint) {};
     
    %FLL Blocks
    \node [block, right of=RotatorPickoffPoint,node distance=3cm] (DiscriminatorFLL) {$K_{DFLL}$};
    
    %Integrator
    \node [input,right of=DiscriminatorFLL] (Point1FLL) {};
    \node [block, below right of=Point1FLL,node distance=2cm] (Delay1FLL){$Z^{-1}$};
    \node [sum, right of=Point1FLL,node distance=3cm] (sumIntegrator1FLL) {\Large$+$};
    
    \node [block, right of=sumIntegrator1FLL,node distance=2cm] (SamplingTime){$\frac{1}{T}$};  
    
    %Proportional FLL 
    \node [block, below of = DiscriminatorFLL,node distance =7cm](K1FLL){$K_{1}$};
    \node [input, below of = K1FLL,node distance = 2cm](Point2FLL){};
    
    %Integrator FLL
    \node [sum,right of = Point2FLL,node distance =3cm] (sumIntegrator2FLL){\Large$+$};
    \node [block, below left of=sumIntegrator2FLL,node distance=2cm] (Delay2FLL){$Z^{-1}$};
    
    \node [block, left of = Point2FLL](K2FLL){$K_{2}$};
    \node [sum, left of=K1FLL,node distance=2cm] (sum1FLL) {\Large$+$};
    \node [input,right of=K1FLL,node distance = 7cm] (PickoffPointFLL) {};
    
    
    
    %PLL Blocks
    \node [block, above of=DiscriminatorFLL] (DiscriminatorPLL) {$K_{DPLL}$};
    
    %Differentiator PLL
    \node [sum, below of = Point2FLL,node distance =3cm](sumDifferentiator1PLL) {\Large$+$};
    \node [input, right of = sumDifferentiator1PLL,node distance = 3cm] (Point2PLL) {};
    \node [block, below right of=sumDifferentiator1PLL,node distance=2cm] (Delay2PLL){$Z^{-1}$};
    
    \node [input, right of = Point2PLL,node distance = 5cm] (PLLPickoffPoint1) {};
    
    
    
    %Proportional PLL
    \node [block, left of = sumDifferentiator1PLL](K1PLL){$K_{3}$};
    \node [block, below of = sumDifferentiator1PLL,node distance = 3cm](K2PLL){$K_{4}$};
    
    
    \node [input, right of = K2PLL,node distance = 8cm] (PLLPickoffPoint2) {};
    
    %Intergral PLL
    \node [input, below of = K2PLL] (Point1PLL) {};
    \node [sum, right of = Point1PLL,node distance =3cm](sumIntegrator1PLL){\Large$+$};
    \node [block, below right of=Point1PLL,node distance=2cm] (Delay1PLL){$Z^{-1}$};
    
    
    \node [block, left of =Point1PLL](K3PLL){$K_{5}$};
    \node [sum, left of=K2PLL,node distance=2cm] (sum1PLL) {\Large$+$};

    %Common
    \node [output, right of=DiscriminatorPLL, node distance=8cm] (output) {};
    


    \node [sum, left of = sum1FLL,node distance =3cm](sumCommon){\Large$+$};
    
    
    \node [input, above of = sumCommon,node distance=3cm](PointNCO){};
    \node [block, above left of = sumCommon](delayNCO){\Large$Z^{-1}$};
    
    \node [block, above of = PointNCO](NCO){NCO};
    
    %\node [block, above left of = PointNCO](delayNCO2){\Large$Z^{-1}$};
    
    
    
    \draw [->] (rinput) -- node{$\phi_{Input}$}(Rotator);
    \draw [->] (PointNCO) -- node{} (NCO);
    \draw [->] (NCO) -- node{$\phi_{NCO}$} node[pos=0.90] {$-$} (Rotator);
    
    
    
    \draw [->] (PointNCO) -| node{} (delayNCO);
    \draw [->] (delayNCO) |- node{} (sumCommon);
    
    %\draw [->] (PointNCO) -- node{} (Rotator);
    
    
    %FLL Connections
    \draw [-] (Rotator) -- node{$\phi_{Error}$} (RotatorPickoffPoint);
    
    \draw [->] (RotatorPickoffPoint) -- node{} (DiscriminatorFLL);
    
    
    \draw [->] (DiscriminatorFLL) -- node{} (sumIntegrator1FLL);
    
    %FLL Discriminator Integrator
    \draw [->] (Point1FLL) |- node{} (Delay1FLL);
    \draw [->] (Delay1FLL) -| node[pos=0.95]{$-$} (sumIntegrator1FLL);
    \draw [->] (sumIntegrator1FLL) -- node{} (SamplingTime);
    
    
    %Loop Filter Integrator
    \draw [->] (Delay2FLL) -| node{} (sumIntegrator2FLL);
    \draw [->] (Point2FLL) |- node{} (Delay2FLL);
    
    \draw [-] (SamplingTime) -- node{} (PickoffPointFLL);
    \draw [->] (PickoffPointFLL) -- node{} (K1FLL);
    \draw [->] (PickoffPointFLL) |- node{} (sumIntegrator2FLL);
    
    %\draw [->] (SamplingTime) |- node{} (K1FLL);
    %\draw [->] (SamplingTime) |- node{} (sumIntegrator2FLL);


    \draw [->] (sumIntegrator2FLL) -- node{} (K2FLL);
    
    \draw [->] (K1FLL) -- node{} (sum1FLL);
    \draw [->] (K2FLL) -- node{} (sum1FLL);
    \draw [->] (sum1FLL) -- node{} (sumCommon);
    
    
    %PLL Connections
    \draw [->] (RotatorPickoffPoint) |- node[]{} (DiscriminatorPLL);
    
    \draw [-] (DiscriminatorPLL) -- node [name=y] {}(output);
    
    \draw [-] (output) -- node{} (PLLPickoffPoint1);
    \draw [->] (PLLPickoffPoint1) -- node{} (sumDifferentiator1PLL);
    \draw [-] (PLLPickoffPoint1) -- node{} (PLLPickoffPoint2);
    
    \draw [->] (PLLPickoffPoint2) -- node{} (K2PLL);
    \draw [->] (PLLPickoffPoint2) |- node{} (sumIntegrator1PLL);
    
    %\draw [->] (output) |- node{} (sumDifferentiator1PLL);
    %\draw [->] (output) |- node{} (K2PLL);
    %\draw [->] (output) |- node{} (sumIntegrator1PLL);
    
    %Integrator
    \draw [->] (sumIntegrator1PLL) -- node{} (K3PLL);
    \draw [->] (Delay1PLL) -| node{} (sumIntegrator1PLL);
    \draw [->] (Point1PLL) |- node{} (Delay1PLL);
    
    
    \draw [->] (sumDifferentiator1PLL) -- node{} (K1PLL);
    
    \draw [->] (Point2PLL) |- node{} (Delay2PLL);
    \draw [->] (Delay2PLL) -| node[pos=0.95]{$-$}  (sumDifferentiator1PLL);
    
    
    
    \draw [->] (K1PLL) -- node{} (sum1PLL);
    \draw [->] (K2PLL) -- node{} (sum1PLL);
    \draw [->] (K3PLL) -- node{} (sum1PLL);
    
    \draw [->] (sum1PLL) -| node{} (sumCommon);
    \draw [-] (sumCommon) -| node{} (PointNCO);
    
    
    %% Boxing and labelling noise shapers
	%\draw [color=gray,thick](-0.5,-3) rectangle (9,1);
	%\node at (-0.5,1) [above=5mm, right=0mm] {\textsc{first-order noise shaper}};
	%\draw [color=gray,thick](-0.5,-9) rectangle (12.5,-5);
	%\node at (0,0) {\textsc{second-order noise shaper}};
    
   
    
    \end{tikzpicture}
\caption{A digital model of the FLL and PLL tracking loops.} \label{fig:StateD_Digital}
\end{figure}


\tikzset{
block/.style = {draw, fill=white, rectangle, minimum height=3em, minimum width=3em},
tmp/.style  = {coordinate}, 
sum/.style= {draw, fill=white, circle, node distance=1cm},
input/.style = {coordinate},
output/.style= {coordinate},
pinstyle/.style = {pin edge={to-,thin,black}
}
}

\begin{figure}[!htb]
\centering
\begin{tikzpicture}[auto, node distance=2cm,>=latex']
    \node [input, name=rinput] (rinput) {};
    \node [sum, right of=rinput,node distance = 2
    cm](Rotator){\Large$+$};
    \node [input, right of = Rotator] (RotatorPickoffPoint) {};
     
    
    
    %FLL Stuff
    \node [block, right of=Rotator,node distance=4cm] (DiscriminatorFLL){$S$};
    
    \node [block, below of = DiscriminatorFLL,node distance =8cm](K1FLL){$K_{1}$};
    \node [block, below of = K1FLL](IntegratorFLL){$\frac{1}{S}$};
    \node [block, left of = IntegratorFLL](K2FLL){$K_{2}$};
    \node [sum, left of=K1FLL,node distance=2cm] (sum1FLL) {\Large$+$};
    \node [sum, left of = sum1FLL,node distance =2cm](sumCommon){\Large$+$};
    
    \node [block, above of = sumCommon](VCOIntegrator){$\frac{1}{S}$};
    \node [block, above of = VCOIntegrator](VCOGain){$K_{VCO}$};
    \node [block, above of = VCOGain](VCO){$\frac{1}{S}$};
    
    
    \node [input, right of = K1FLL] (FLLPickoffPoint) {};
    
    
    
    %PLL Stuff
    \node [block, above of=DiscriminatorFLL] (DiscriminatorPLL) {$1$};
    \node [block, below of = IntegratorFLL](diffPLL){$S$};
    \node [block, left of = diffPLL](K1PLL){$K_{3}$};
    \node [block, below of = diffPLL](K2PLL){$K_{4}$};
    \node [block, below of = K2PLL](IntegratorPLL){$\frac{1}{S}$};
    \node [block, left of = IntegratorPLL](K3PLL){$K_{5}$};
    \node [sum, left of=K2PLL,node distance=2cm] (sum1PLL) {\Large$+$};
    \node [output, right of=DiscriminatorPLL, node distance=4cm] (output) {};
     
    
    \node [input, right of = diffPLL,node distance =4cm] (PLLPickoffPoint1) {};
    \node [input, right of = K2PLL,node distance =4cm] (PLLPickoffPoint2) {};
    
    
    %Common
    \draw [->] (rinput) -- node{$\phi_{Input}$}(Rotator);
    
    %FLL Stuff
    \draw [-] (Rotator) -- node{$\phi_{Error}$} (RotatorPickoffPoint);
    \draw [->] (RotatorPickoffPoint) -- node{} (DiscriminatorFLL);
    
    
    \draw [->] (DiscriminatorFLL)  -| node{} (FLLPickoffPoint);
    \draw [->] (FLLPickoffPoint) -- node{} (K1FLL);
    \draw [->] (FLLPickoffPoint) |- node{} (IntegratorFLL);
    \draw [->] (IntegratorFLL) -- node{} (K2FLL);
    
    \draw [->] (K1FLL) -- node{} (sum1FLL);
    \draw [->] (K2FLL) -- node{} (sum1FLL);
    \draw [->] (sum1FLL) -- node{} (sumCommon);
    
    
    %PLL Stuff
    \draw [->] (RotatorPickoffPoint) |- node{} (DiscriminatorPLL);
    
    \draw [-] (DiscriminatorPLL) -- node [name=y] {}(output);
    
    \draw [-] (output) -- node{} (PLLPickoffPoint1);
    \draw [-] (PLLPickoffPoint1) -- node{} (PLLPickoffPoint2);
    \draw [->] (PLLPickoffPoint1) -- node{} (diffPLL);
    \draw [->] (PLLPickoffPoint2) -- node{} (K2PLL);
    \draw [->] (PLLPickoffPoint2) |- node{} (IntegratorPLL);
   

    \draw [->] (IntegratorPLL) -- node{} (K3PLL);
    \draw [->] (diffPLL) -- node{} (K1PLL);
    
    \draw [->] (K1PLL) -- node{} (sum1PLL);
    \draw [->] (K2PLL) -- node{} (sum1PLL);
    \draw [->] (K3PLL) -- node{} (sum1PLL);
    
    \draw [->] (sum1PLL) -| node{} (sumCommon);
    \draw [->] (sumCommon) -- node{} (VCOIntegrator);
    \draw [->] (VCOIntegrator) -- node{} (VCOGain);
    \draw [->] (VCOGain) -- node{} (VCO);
    \draw [->] (VCO) -- node{$\phi_{VCO}$} node[pos=0.90] {$-$} (Rotator);
    
    

    \end{tikzpicture}
\caption{A Laplace domain model of state D.} \label{fig:StateD_Laplace}
\end{figure}


\subsection{Open loop transfer function}
\begin{comment}
(18, 1)
(1, array([ 0.        ,  0.01131368,  0.00022784]))
(18, 1)
(2, array([ 55.06692161,   4.63278117,   0.77306969]))
\end{comment}

\begin{equation}
(S \times \frac{1}{T} \times (K_1 + \frac{1}{S} \times K_2) + K_3 \times S + K_4 + \frac{1}{S} \times K_5) \times \frac{1}{S} \times K_{VCO} \times \frac{1}{S}
\end{equation}

\begin{equation}
(\frac{K_1 S + K_2}{T} + K_3  S + K_4 + \frac{K_5}{S} ) \times \frac{K_{VCO}}{S^2}
\end{equation}

\subsection{Closed loop transfer function}

\begin{comment}
PLL/FLL S^2 Ratio = 19.4
PLL/FLL S Ratio = 81.3339
\end{comment}

\begin{equation}
\frac{1}{1+(\frac{K_1 S + K_2}{T} + K_3  S +K_4  + \frac{K_5}{S}  \times \frac{K_{VCO}}{S^2}}
\end{equation}


\begin{equation}
\frac{S^3}{S^3 + K_{VCO}(\frac{K_1 S^2 + K_2 S}{T} + K_3 S^2 + K_4 S + K_5)}
\end{equation}

\begin{framed}
\begin{align}
FLL_{BW}&=1\\
PLL_{BW}&=18\\
K_1 &=  0.01131\\
K_2 &=  0.0002278\\
K_3 &= 55.07\\
K_4 &= 4.633\\
K_5 &= 0.7731
\end{align}
\end{framed}

\begin{equation}
\frac{S^3}{S^3 +2.828 S^2 + 0.05696 S + 55.07 S^2 + 4.633 S + 0.7731}
\end{equation}

\begin{equation}
\frac{S^3}{S^3 +57.90 S^2 + 4.690 S +0.7731}
\end{equation}

\subsection{Step response}

A step input is $\frac{1}{S}$

\begin{equation}
\frac{S^2}{S^3 +57.90 S^2 + 4.690 S +0.7731}
\end{equation}

\begin{equation}
\frac{-0.001397 S - 0.0002316}{S^2 + 0.08087 S + 0.01337} + \frac{1.001}{S + 57.81}
\end{equation}

\subsection{Ramp response}

\begin{equation}
\frac{S}{S^3 +57.90 S^2 + 4.690 S +0.7731}
\end{equation}

\begin{equation}
\frac{0.01732 S + 0.000004006}{S^2 +0.08089 S +0.01337} - \frac{0.01732}{S + 57.81}
\end{equation}

\subsection{Parabolic response}

\begin{equation}
\frac{1}{S^3 +57.90 S^2 + 4.690 S +0.7731}
\end{equation}

\begin{equation}
\frac{-0.0002996 S + 0.01730}{S^2 + 0.08089 S + 0.01337}+\frac{0.0002996}{S+ 57.81}
\end{equation}


