\chapter{Discriminators \& sampling}
\label{ch:Discriminators}

The role of the discriminator is to determine the error (frequency or phase), based on current an previous I \& Q samples. This is computed every second incoherent dump (8ms), because a pair of samples is required to compute frequency. 



\tikzset{
block/.style = {draw, fill=white, rectangle, minimum height=3em, minimum width=3em},
tmp/.style  = {coordinate}, 
sum/.style= {draw, fill=white, circle, node distance=1cm},
input/.style = {coordinate},
output/.style= {coordinate},
pinstyle/.style = {pin edge={to-,thin,black}
}
}

\begin{figure}[!htb]
\centering
\begin{tikzpicture}[auto, node distance=2cm,>=latex']

%\draw[dotted](-3,0.25) node {Coherent samples};
%\draw[dotted](-3,-1.5) node {Incoherent integration};
%\draw[dotted](-3,-3.5) node {Frequencies};

    

\foreach \x in {0,...,31}{
    \draw [fill=orange] (0.5*\x,0) rectangle (0.5*\x+0.5,0.5);
}

\foreach \x in {0,...,7}{
    \draw [fill=blue] (2*\x,-1) rectangle (2*\x+2,-2);
}

\foreach \x in {0,...,3}{
    \draw [fill=teal] (4*\x,-3) rectangle (4*\x+4,-4);
}

\foreach \x in {0,...,7}{
\draw [decorate,decoration={brace,amplitude=0.4cm,mirror},xshift=0,yshift=0pt]
(2*\x,0) -- (2*\x+2,0) node [black,midway]{};
}

\foreach \x in {0,...,7}{
    \draw[->] (2*\x+1,-0.4) -- (2*\x+1,-1);
}

\foreach \x in {0,...,3}{
    \draw [decorate,decoration={brace,amplitude=0.4cm,mirror},xshift=0,yshift=0pt]
(4*\x,-2) -- (4*\x+4,-2) node [black,midway]{};
}

\foreach \x in {0,...,3}{
    \draw[->] (4*\x+2,-2.4) -- (4*\x+2,-3);
}

\end{tikzpicture}
\caption{An overview of the sampling process.1 ms long coherently integrated samples (orange)  are produced by the hardware correlator.Four of these are coherently integrated together to generate the inputs to the tracking loop (blue). The FLL and PLL discriminator both take a pair of 4ms long samples to produce an estimate of the phase/frequency (teal).}
\label{fig:SamplingSchemeA}
\end{figure}






\tikzset{
block/.style = {draw, fill=white, rectangle, minimum height=3em, minimum width=3em},
tmp/.style  = {coordinate}, 
sum/.style= {draw, fill=white, circle, node distance=1cm},
input/.style = {coordinate},
output/.style= {coordinate},
pinstyle/.style = {pin edge={to-,thin,black}
}
}

\begin{figure}[!htb]
\centering
\begin{tikzpicture}[auto, node distance=2cm,>=latex']

%\draw[dotted](-3,0.25) node {Coherent samples};
%\draw[dotted](-3,-1.5) node {Incoherent integration};
%\draw[dotted](-3,-3.5) node {Frequencies};

\foreach \x in {0,...,31}{
    \draw [fill=orange] (0.5*\x,0) rectangle (0.5*\x+0.5,0.5);
}

\foreach \x in {0,...,7}{
    \draw [fill=blue] (2*\x,-1) rectangle (2*\x+2,-2);
}


\foreach \x in {0,...,7}{
    \draw [fill=teal] (2*\x,-3) rectangle (2*\x+2,-4);
}


\foreach \x in {0,...,7}{
    \draw [decorate,decoration={brace,amplitude=0.4cm,mirror},xshift=0,yshift=0pt]
(2*\x,0) -- (2*\x+2,0) node [black,midway]{};
    \draw[->] (2*\x+1,-0.4) -- (2*\x+1,-1);
}


\foreach \x in {0,...,3}{
    \draw [decorate,decoration={brace,amplitude=0.5cm,mirror},xshift=0,yshift=0pt]
(4*\x,-2) -- (4*\x+4,-2) node [black,midway]{};
    \draw[->] (4*\x+2,-2.5) -- (4*\x+1,-3);
}

\foreach \x in {0,...,2}{
    \draw [red,decorate,decoration={brace,amplitude=0.3cm,mirror},xshift=0,yshift=0pt]
(4*\x+2,-2) -- (4*\x+6,-2) node [black,midway]{};
    \draw[->,red] (4*\x+4,-2.3) -- (4*\x+3,-3);
}

\end{tikzpicture}
\caption{As compared to the sampling process outlined in figure \ref{fig:SamplingSchemeA}, this scheme provides frequency and phase estimates every 4ms.}
\label{fig:SamplingB}
\end{figure}






\tikzset{
block/.style = {draw, fill=white, rectangle, minimum height=3em, minimum width=3em},
tmp/.style  = {coordinate}, 
sum/.style= {draw, fill=white, circle, node distance=1cm},
input/.style = {coordinate},
output/.style= {coordinate},
pinstyle/.style = {pin edge={to-,thin,black}
}
}

\begin{figure}[!htb]
\centering
\begin{tikzpicture}[auto, node distance=2cm,>=latex']

%\draw[dotted](-3,0.25) node {Coherent samples};
%\draw[dotted](-3,-1.5) node {Incoherent integration};
%\draw[dotted](-3,-3.5) node {Frequencies};


\draw[->] (1,1) -- (1,0.5);
\draw[->] (3,1) -- (3,0.5);
\node at (1,1.2) {$\phi_1$};
\node at (3,1.2) {$\phi_2$};


\foreach \x in {0,...,7}{
    \draw [fill=orange] (0.5*\x,0) rectangle (0.5*\x+0.5,0.5);
}


\foreach \x in {0,...,7}{
    \draw [fill=orange] (0.5*\x+9,0) rectangle (0.5*\x+0.5+9,0.5);
}


\node at (4.5,0) {\ldots};


\foreach \x in {0,...,1}{
    \draw [fill=blue] (2*\x,-1) rectangle (2*\x+2,-2);
}

\draw [fill=teal] (0,-3) rectangle (4,-4);


\foreach \x in {0,...,1}{
    \draw [decorate,decoration={brace,amplitude=0.4cm,mirror},xshift=0,yshift=0pt]
(2*\x,0) -- (2*\x+2,0) node [black,midway]{};
    \draw[->] (2*\x+1,-0.4) -- (2*\x+1,-1);
}


\draw [decorate,decoration={brace,amplitude=0.5cm,mirror},xshift=0,yshift=0pt]
(0,-2) -- (4,-2) node [black,midway]{};
\draw[->] (2,-2.5) -- (2,-3);


\draw[->] (4,-4) -- (4,-5);
\draw [fill=purple] (4,-5) rectangle (5.5,-6);
\node at (4.75,-5.5) {Jitter};


\draw[->] (5.5,-6) -- (5.5,-7);
\draw [fill=yellow] (5.5,-7) rectangle (7.5,-8);
\node at (6.5,-7.25) {Tracking};
\node at (6.5,-7.75) {Loop};


\draw[->] (7.5,-8) -- (7.5,-9);
\draw [fill=purple] (7.5,-9) rectangle (9,-10);
\node at (8.25,-9.5) {Jitter};

\draw[->] (9,-9) -- (9,0);
\node at (10.5,-0.5) {Update NCO};


\draw [decorate,decoration={brace,amplitude=0.4cm},xshift=0,yshift=0pt]
(3,0.5) -- (9,0.5) node [black,midway]{};

%\draw[->] (9,1.5) -- (9,0.5);
\node at (6,1.2) {Delay};


\end{tikzpicture}
\caption{Once a pair of 4ms coherently integrated samples has been collected, then the tracking loops can compute a new value for the NCO. During this process, delays accrue due to the interrupt driven nature of the processor, as well as processing time for the tracking loops.}
\label{fig:SamplingJitter}
\end{figure}







\subsection{FLL discriminator}
The role of the FLL discriminator is to convert pairs of phase measurements into frequency. A key statistic for the FLL discriminator is it's capture range. This can be computed as 
$\frac{-1}{2T} < f < \frac{1}{2T}$
, where $T = 0.004S$

Hence the FLL has a locking range of $\pm 125Hz$. 

If the frequency of the incoming signal is outside this range, then a false lock will occur due to aliasing. This can be seen in figure(Insert picture of aliasing). 

The FLL discriminator which is described in Kaplan uses the dot and cross product of the previous two samples.

\begin{align*}
\text{Dot} &= I_{k}\times I_{k-1} + Q_{k}\times Q_{k-1}\\
\text{Cross} &= Q_{k} \times I_{k-1} - I_{k} \times K_{k-1}\\
\text{DeltaPhase} &= ATan2(Cross,Dot)
\end{align*}
